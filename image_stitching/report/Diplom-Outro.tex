\chapter*{Висновки}
%\addcontentsline{toc}{chapter}{Висновки}

В дисертаційній роботі було досліджено актуальні на сьогоднішній день задачі комп'ютерного зору, зокрема співставлення зображень та розпізнавання об'єктів на них. Було розглянуто різноманітні підходи та проаналізовано їх переваги та недоліки. Додаткове дослідження було проведено відносно підходів, що реалізують співставлення зображень шляхом пошуку локальних особливостей. Було проведено додаткове порівняння ефективності застосування алгоиртмів SURF, SIFT, ASIFT до задач порівняння статичних, а особливо динамічних відеозображень. Було запропоновано новий метод покращення результатів застосування відомих алгоритмів співставлення статичних зображень. Аналіз отриманих результатів дозволяє зробити наступні висновки:

\begin{enumerate}
\item При порівнянні ефективності застосування алгоритмів за показниками кількості знайдених та співставлених ключових точок, до обробки окремих кадрів відеопотоків було підтверджено перевагу підходу, запропонованого в рамках ASIFT над іншими алгоритмами. З цього було зроблено висновки, що доцільно проводити подальші дослідження, базуючись на результатах, отриманих при використанні ASIFT.
\item Аналіз ефективності роботи як алгоритму ASIFT, так і SIFT та SURF показав, що підхід безпосереднього порівняння відповідних кадрів відео, як окремих статичних зображень, хоч і має місце, але показує незадовільні результати точності та стійкості до зашумленості даних, наприклад у задачах трекінгу та побудови панорамних зображень.
\item Зважаючи на незадовільність безпосереднього застосування вихідних алгоритмів до окремих відеокадрів, було прийнято рішення про розробку методу покращення результатів, зокрема зменшення впливу шуму та випадкових збурень на результаті вирішення таких задач, як трекінг та об'єднання зображень, за рахунок використання додаткової інформації, що може бути отримана з відеопотоку.
\item Було розроблено метод застосування алгоритмів співставлення зображень, що дійсно дозволяє та суттєво покращує результати їх застосування саме до динамічних послідовностей зображень, де між зображеннями послідовних кадрів існує суттєва кореляція. Запропонований метод полягає в обчисленні матриці геометричного перетворення між досліджуваними зображеннями в комплексі з застосуванням додаткової фільтрації за принципом ковзного середнього. 
\item Спроектовано та реалізовано програмний засіб для реалізації запропонованого алгоритму. 
\item Проведено дослідження впливу запропонованого методу на результати вирішення задач трекінгу та об'єднання зображень, що показало суттєве покращення результатів, зокрема зменшення коливань матриць геометричного перетворення з плином кадрів, спричинених випадковими збуреннями, присутніми внаслідок зашумленості вихідних даних, пов'язаних з недосконалістю вимірювальних пристроїв.
\end{enumerate}

\chapter*{Рекомендації}

Було показано, що запропонований метод дає суттєве покращення результатів застосування відомих алгоритмів SIFT, ASIFT до даних з відеопотоків, а тому має сенс застосовувати його в таких задач. 

Тим не менш, проведений аналіз показав, що запропонований підхід має і деякі обмеження, зумовлені вимогами використаних алгоритмів SIFT, а особливо ASIFT до обчислювальних ресурсів. 

Хоча на сьогоднішній день і існує багато напрямів, за якими можна оптимізувати ресурсні потреби роботи досліджених алгоритмів, такі як об'єми необхідної пам'яті, потужність обчислювальних процесорів та, власне, необхідний час, ці дослідження виходять за рамки даної роботи та пропонуються для розгляду в подальшому.

