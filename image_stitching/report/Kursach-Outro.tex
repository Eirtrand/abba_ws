\chapter{Висновки}

В цій роботі було досліджено два передові алгоритми розпізнання цифрових зображень, що на сьогоднішній день є дуже перспективними. До сфер, в яких вони можуть знайти використання, належать відео спостереження, автоматична класифікація цифрових зображень, взаємодія людини з комп'ютером (HCI) та ін. 

Було зазначено порівняльну характеристику описаних алгоритмів на декількох наборах зображень, що показали особливості роботи кожного з них на характерних класах досліджуваних вхідних даних. З результатів порівняння видно, що алгоритм ASIFT дійсно є суттєвим покращенням SIFT, що надає йому необхідну властивість афінної інваріантності, якої бракувало оригінальному алгоритму. Крім того, з результатів порівняння видно, що ASIFT краще працює і в умовах не високих афінних спотворень. Тому, якщо ускладнення обчислень приблизно в два рази не є суттєвим, то для задач розпізнавання зображень слід надавати перевагу алгоритму ASIFT, а не SIFT.

Оскільки схеми їх роботи є відкритими, то всі дослідники мають можливість експериментувати з різними особливостями конкретних реалізацій, тому маємо широкий простір для подальших досліджень.
