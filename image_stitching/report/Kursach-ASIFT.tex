\chapter{ASIFT}
Якщо об'єкт на зображенні має чіткі, неперервні границі, то його зображення, отримане камерами з різних позицій та кутів огляду, може суттєво змінюватися. Ці трансформації легко наближати афінними перетвореннями площини зображення.

Тим не менш, проблема співставлення різних зображень одного об'єкта, отриманих під різними кутами нахилу, залишається і до нині. Одним з підходів вирішення цієї проблеми може бути побудова дескрипторів для кожної ключової точки зображення, інваріантних відносно афінних перетворень, але таких алгоритмів на сьогоднішній день не існує. 

Алгоритм ASIFT~\cite{Morel2009} (Affine-SIFT) використовує інший підхід. Замість побудови афінно-інваріантних дескрипторів напряму, автором було запропоновано симулювати різні можливі кути нахилу спостерігача (камери) у двох площинах. Так, варіюючи довготу та широту положення камери до вихідного зображення, ми маємо ширші можливості для порівняння двох зображень, що відрізняються кутом огляду, без значного збільшення складності обчислень. Дворозмірна схема роботи ASIFT дозволяє отримати результати, загалом, кращі, ніж SIFT, ускладнюючи обчислення не більше, ніж в два рази. 

Для опису алгоритму використовується нове поняття, \textit{відносний нахил} (\textit{transition tilt}), що описує рівень спотворення зображення при переході від одного кута огляду до іншого. 

\section{Модель отримання вихідного зображення}
Робота алгоритму ASIFT базується на наступній моделі отримання цифрового зображення (рис~\ref{fig:camera-model}).

\coolwfigure{camera-model}{Модель цифрової камери: $\textbf{u}=\textbf{S}_1G_1\mathcal{A}u_0$. $\mathcal{A}$ - перетворення плоскої проекції (гомографія). $G_1$ - згладжуючий гаусівський фільтр. $\textbf{S}_1$ - вибірка CCD матрицею.}{fig:camera-model}{0.5\linewidth}

Як зображено на рис~\ref{fig:camera-model}, отримання зображення плоского об'єкту може бути описане наступним чином:

\begin{equation}
  \textbf{u}=\textbf{S}_1G_1A\mathcal{T}u_0
\end{equation}

де $\textbf{u}$ -- отримане цифрове зображення, а $u_0$ - нескінченний фронтальний вигляд плоского об'єкта. $\mathcal{T}$ та $A$ -- відповідно зміщення та плоска проекція, пов'язані з переміщенням камери. $G_1$ -- гаусівська згортка, що моделює оптичне розмиття, а $\textbf{S}_1$ -- оператор дискретизації зображення на сітці з кроком 1. 

Ми апроксимуємо оператор $A$, що насправді дає зображення в перспективі, афінним перетворенням для спрощення обчислень. Тим не менш дана апроксимація є досить точною, оскільки в подальшому ми розглядаємо лише локальні особливості зображення, а при розгляданні малих його областей перспектива майже повністю співпадає з плоским афінним перетворенням. Також аналогічний результат видно з розкладання будь-якого неперервного перетворення в ряд Тейлора. В кожній точці перетворення може бути описано афінним оператором. Таким чином, всі локальні ефекти перспективи можуть бути змодельовані як $u(x,y) \rightarrow u(ax+by+e, cx+dy+f)$ в кожній точці.

Морелом було доведено~\cite{Morel2009} наступну теорему. 

\textsc{Теорема.}
\textit{
Будь-який лінійний оператор 
$A=\begin{bmatrix}a&b \\ c&d\end{bmatrix}$ з $detA > 0$, що не є тотожною матрицею може бути представлений єдиним чином у вигляді розкладу:
  \begin{equation}
    \label{eq:asift-decomposition}
    A = H_\lambda R_1(\psi)T_tR_2(\phi) =     
    \lambda \begin{bmatrix} 
      \cos \psi & -\sin\psi \\
      \sin\psi  & \cos\psi
    \end{bmatrix}
    \begin{bmatrix}
      t & 0 \\
      0 & 1 
    \end{bmatrix}
    \begin{bmatrix} 
      \cos \phi & -\sin\phi \\
      \sin\phi  & \cos\phi
    \end{bmatrix}
  \end{equation}
  де $\lambda>0$, $\lambda t$ - визначник матриці $A$, $R_i$ - повороти, $\phi \in \left[0,\pi\right)$, та $T_t$ - нахил, що є діагональною матрицею з власними числами $t>1$ та 1.
}

Отриманий результат є пов'язаним з принципом SVD-розкладу. 

\coolwfigure{asift-decomposition}{Геометрична інтерпретація розкладу~\ref{eq:asift-decomposition}.}{fig:asift-decomposition}{0.2\linewidth}

На рисунку~\ref{fig:asift-decomposition} зображено графічну інтерпретацію наведеного вище розкладу. $\phi$ та $\theta= \arccos1/t$ - кути огляду, $\psi$ параметризує обертання камери, а $\lambda$ відповідає масштабу. Площина, що містить нормаль і оптичну вісь, утворює кут $\phi$ з деякою фіксованою вертикальною площиною. Цей кут називають довготою. Оптична вісь утворює з нормаллю кут $\theta$, який називають широтою. 

\section{Відносний нахил}

Параметр $t$ в \ref{eq:asift-decomposition} називається \textit{абсолютним нахилом}, оскільки описує відхилення фактичної точки зору від фронтального вигляду. Для виміру спотворень між двома зображеннями, що відрізняються кутом нахилу, також вводять поняття \textit{відносного нахилу}.

\textsc{Означення}. Розглянемо два вигляди плоского зображення, $u_1(x,y) = u(A(x,y))$ та $u_2(x,y)=u(B(x,y))$ де $A$ та $B$ -- два такі оператори, що $BA^{-1} \ne I$. Використовуючи позначення з \ref{eq:asift-decomposition}, будемо називати відповідно \textit{відносним нахилом} $\tau(u_1,u_2)$ та \textit{відносним поворотом} $\psi(u_1,u_2)$ єдині значення параметрів такі, що 
\[
  BA^{-1} = H_\lambda R_1(\phi)T_\tau R_2(\psi).
\]

\section{Алгоритм ASIFT}
\label{sec:algo-asift}
Алгоритм ASIFT складається з наступних кроків:

\begin{enumerate}
    \item Для кожного з вхідних зображень симулюються всі можливі спотворення, викликані відхиленням оптичної осі камери від фронтального положення. Ці спотворення описуються двома параметрами: довготою $\psi$ та широтою $\phi$. Після обертання на кут $\phi$, зображення нахиляють з параметром $t = \left| \frac{1}{\cos\theta}\right|$ (нахил на $t$ в напрямку $x$ є операцією $u(x,y) \rightarrow u(tx,y)$). Для цифрових зображень поворот реалізується направленим $t$-сабсемплінгом і вимагає попереднього згладжування згорткою з гаусівською функцією з дисперсією $c\sqrt{t^2-1}$. Значення $c=0.8$ було вибрано Ловом~\cite{Lowe2004}.
    \item Ці повороти і нахили проводяться для скінченної і невеликої кількості кутів довготи і широти. При цьому кроки дискретизації для цих параметрів вибираються так, щоб для кожного можливого значення цих параметрів поблизу знаходилася точка отриманої сітки дискретизації.
    \item Всі отримані симуляцією зображення порівнюються за допомогою деякого іншого алгоритму (SIFT).
\end{enumerate}

\coolwfigure{asift-sampling}{Дискретизація параметрів $\theta=\arccos 1/t$ та $\phi$. Зліва: зображення в перспективі півкулі спостережень (показані лише $t=2,2\sqrt{2},4$). Справа: вигляд з зеніту півсфери спостереження.}{fig:asift-sampling}{0.8\linewidth}

Дискретизація параметрів спотворення проводиться наступним чином:
\begin{itemize}
    \item Широта $\theta$ вибирається так, що відповідні нахили утворюються геометричну прогресію $1, a, a^2, \ldots,a^n$, з $a>1$. Значення $a = \sqrt{2}$ є добрим компромісом між точністю і розрідженістю. 
    \item Довгота $\phi$ для кожного значення нахилу утворює арифметичну прогресію $0, b/t, \ldots, kb/t$, де значення $b \simeq 72^{\degree}$ також є раціональним компромісом, а $k$ -- найбільше ціле число таке, що $kb/t < 180^{\degree}$.
\end{itemize}


\section{Оптимізація двома розмірами}

Для прискорення роботи алгоритму ASIFT, використовується схема роботи з зображеннями двох розмірів. Спочатку, алгоритм, описаний в \ref{sec:algo-asift}, застосовують до зменшених варіантів досліджуваних зображень. У випадку успішного співставлення малих зображень, процедура вибирає отримані значення афінного перетворення та, застосовуючи їх до повнорозмірних зображень, порівнює їх за допомогою алгоритму SIFT. Дворозмірний метод можна описати наступними кроками:

\begin{enumerate}
  \item Зменшення досліджуваного та еталонного зображення в K$\times$K разів: $\mathbf{u}' = \mathbf{S}_K\mathbf{G}_K\mathbf{u}$ та $\mathbf{v}' = \mathbf{S}_K\mathbf{G}_K\mathbf{v}$, де $\mathbf{G}_K$ -- згладжуючий гаусівський дискретний фільтр, а $\textbf{S}_K$ -- оператор зменшення розміру зображення (subsampling) в K$\times$K разів
  \item Застосування алгоритму ASIFT, описаного в \ref{sec:algo-asift}, до зменшених зображень $u'$ та $v'$.
  \item Вибрати $M$ афінних перетворень, що дають найбільшу кількість збігів на зменшених зображеннях $u'$ та $v'$.
  \item Застосувати ASIFT до  $u$ та $v$, але симулювати лише $M$ афінних перетворень.

\end{enumerate}
